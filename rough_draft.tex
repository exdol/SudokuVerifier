\documentclass[11pt]{article}

\usepackage{float}
\usepackage[T1]{fontenc}
\usepackage{graphicx}
\usepackage[utf8]{inputenc}
\usepackage[backend=biber, sorting=none]{biblatex}
\addbibresource{bibliography-biblatex.bib}
\usepackage[paperheight=27.94cm,paperwidth=21.59cm,left=2.54cm,right=2.54cm,top=2.54cm,bottom=2.54cm]{geometry}

\setlength\parindent{0pt}
\renewcommand{\arraystretch}{1.3}

\title{Sudoku Solver}

\begin{document}
\maketitle

\vspace{2\baselineskip}
Exdol Davy

Douglas Glover

Kashyap Rana

Matthew Singh

Miguel Venero

\vspace{1\baselineskip}

\begin{center}
{\Large\textbf{Abstract}}
\end{center}

\vspace{1\baselineskip}
Malformed Sudoku puzzles (or Sudoku puzzles that have more than one solution) are an unlikely occurrence in today's world of computers, but they are not impossible to be produced accidentally (or even on purpose). It has been mathematically proven by Gary McGuire, Bastian Tugemann, and Gilles Civario in 2012 [1] that a sudoku puzzle must provide at least 17 valid clues to contain exactly 1 unique solution. This project will provide further analysis upon this proof, while determining whether a sudoku puzzle can be solved or not.

\begin{center}
{\Large\textbf{Introduction}}
\end{center}

\vspace{1\baselineskip}
Sudoku is a Japanese puzzle game that requires players to fill all of 81 boxes presented in a 9 by 9 grid. The content of each box must be any integer between the interval of 1 through 9 inclusively. The 9 by 9 grid is also divided into 9 blocks (with 3 blocks being in each row and 3 blocks being in each column), each of which contain 9 boxes, therefore also being a 3 by 3 grid. The image below is an example of the 9 by 9 grid, with all 9 of the 3 by 3 gridded blocks.

\vspace{1\baselineskip}
\begin{figure}[H]
\centering
\includegraphics[width=5.11cm,height=4.87cm]{sudoku-blankgrid.png}
\end{figure}

\vspace{1\baselineskip}
Each sudoku puzzle typically starts off with at least 17 clues to be guaranteed to have a valid solution, but can contain more depending on the difficulty level. However, to successfully complete the game, the player must fill all empty boxes, in the 9 by 9 grid, such that each of the 9 blocks contain exactly 9 integers from the interval of 1 through 9 inclusive without repetition (each integer being within its own respective box). In addition to this, the boxes within each column and each row of the 9 by 9 grid must contain an integer from the interval of 1 through 9 inclusive without any duplicates being in each column and row respectively. 

\vspace{1\baselineskip}

\begin{center}
{\Large\textbf{Methodology}}
\end{center}

\vspace{1\baselineskip}
To better understand how to systematically solve a sudoku, we first created a single thread implementation of solving sudoku which was a brute force method where it checks all possible values for each cell and iterates until the grid has been solved.

\vspace{1\baselineskip}
The next step was to come up with a multithreading approach for the problem. At first we thought to have two threads, one for the column and one for the row, then both would work together to find all possible options for each cell to solve the sudoku. This method does not properly use multithreading concepts to their full potential as it takes the single thread approach and assigns threads to specific assignments. We decided that we did not want our project to move in this direction so we halted the implementation of this approach.

\vspace{1\baselineskip}
After some discussions on what multithreading approaches we should take, more research was conducted. To properly solve a grid we dived into understanding how average humans go about solving the sudoku, the pencil and paper algorithm. A human tends to store all possible options for each cell then go cell by cell and decide which values are valid and settle on one value per cell. We attempted to do the same by  first calculating all possible values for each cell then using nine threads, one for each 3x3 grid and solving the grid. 

\begin{center}
{\Large\textbf{References}}
\end{center}

\vspace{1\baselineskip}
[1] McGuire, Gary, et al. “There Is No 16-Clue Sudoku: Solving the Sudoku Minimum Number of Clues Problem.” NASA/ADS, https://ui.adsabs.harvard.edu/abs/2012arXiv1201.0749M/abstract. 


\printbibliography


\end{document}




